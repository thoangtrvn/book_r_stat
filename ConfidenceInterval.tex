
\chapter{Confidence Interval (CI)}
\label{chap:conf-interv-ci}


% \section{Normal distribution}
% \label{sec:normal-distribution}

For a given sample of size $n$, we can estimate the
$100\%\times(1-\alpha)$ confidence interval (CI) of that
sample. Normally, we want to find the CI of the true unknown mean of
the population from which the sample is drawn. 

Clearly, the CI varies from sample to sample. The length of the CI can
give some ideas of the precision of the point estimated,
e.g. $\bar{x}$. The smaller the length of the CI, the better the
sample mean is used to estimate the population mean. 

In addition to finding the CI for the mean - which we need to use
z-test or t-test, we also want to find the CI for the variance - then
we need to use Chi-square distribution.

We may have (Sect.~\ref{sec:binom-distr})
\begin{enumerate}
\item Two-sided confidence interval
\item One-sided confidence interval
\end{enumerate}

The {\bf one-sided CI} for the binomial parameter $p$ using
normal-theory approach 
\begin{itemize}
\item upper one-sided $\ci$ CI is of the form $p> p_1$ with
  \begin{equation}
    \label{eq:55}
    \pr(p > p_1) = 1-\alpha
  \end{equation}
  If $n\hat{p}\hat{q} \ge 5$, then $\hat{p}\sim \text{Norm}(p,pq/n)$ and 
  \begin{equation}
    \label{eq:56}
    p > \hat{p} - z_{1-\alpha} \sqrt{\hat{p}\hat{q}/n}
  \end{equation}
\item lower one-sided $\ci$ CI is
  \begin{equation}
    \label{eq:57}
    \pr(p<p_2) = 1-\alpha
  \end{equation}
  and
  \begin{equation}
    \label{eq:58}
    p < \hat{p} + z_{1-\alpha} \sqrt{\hat{p}\hat{q}/n}
  \end{equation}
\end{itemize}

\begin{framed}
  We use $z_{1-\alpha/2}$ for two-sided CI, but for one-sided CI, we
  need to use $z_{1-\alpha}$. 
\end{framed}
Use the similar methods for the one-sided CI for the mean and variance
of a normal distribution, for the binomial parameter $p$, and for the
Poisson expectation $\mu$ using exact methods.


Go to Chap.~\ref{chap:generate-data}.

\section{Bayesian CI}
\label{sec:bayesian-ci}

\begin{equation}
  \label{eq:67}
  \pr(\mu_1 < \mu < \mu_2) = 1-\alpha
\end{equation}
where
\begin{equation}
  \label{eq:68}
  \begin{split}
    \mu_1 = \bar{x} - z_{1-\alpha/2}\frac{\sigma}{\sqrt{n}} \\
    \mu_2 =  \bar{x} + z_{1-\alpha/2}\frac{\sigma}{\sqrt{n}} \\
  \end{split}
\end{equation}



%%% Local Variables: 
%%% mode: latex
%%% TeX-master: "R_language"
%%% End: 

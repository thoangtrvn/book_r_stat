%%
%% Parallel.tex
%% Login : <hoang-trong@hoang-trong-laptop>
%% Started on  Mon Jun 22 21:51:55 2009 Hoang-Trong Minh Tuan
%% $Id$
%% 
%% Copyright (C) 2009 Hoang-Trong Minh Tuan
%% This program is free software; you can redistribute it and/or modify
%% it under the terms of the GNU General Public License as published by
%% the Free Software Foundation; either version 2 of the License, or
%% (at your option) any later version.
%% 
%% This program is distributed in the hope that it will be useful,
%% but WITHOUT ANY WARRANTY; without even the implied warranty of
%% MERCHANTABILITY or FITNESS FOR A PARTICULAR PURPOSE.  See the
%% GNU General Public License for more details.
%% 
%% You should have received a copy of the GNU General Public License
%% along with this program; if not, write to the Free Software
%% Foundation, Inc., 59 Temple Place, Suite 330, Boston, MA 02111-1307 USA
%%

\chapter{Parallel R simulation}
\label{chap:parall-r-simul}


\url{https://csgillespie.github.io/efficientR/performance.html}

\begin{verbatim}
library("microbenchmark")
library("ggplot2movies")
library("profvis")
library("Rcpp")


library("profvis")
profvis({
  data(movies, package = "ggplot2movies") # Load data
  movies = movies[movies$Comedy == 1,]
  plot(movies$year, movies$rating)
  model = loess(rating ~ year, data = movies) # loess regression line
  j = order(movies$year)
  lines(movies$year[j], model$fitted[j]) # Add line to the plot
})
\end{verbatim}

\section{In Windows}
\label{sec:windows}

\begin{lstlisting}
cl <- makeCluster(2,type="SOCK")
\end{lstlisting}


\section{In Linux}
\label{sec:linux}

purpose-made packages for R : Snow, Rmpi, snowfall,...

\section{serial versions}


When you have a list of repetitive tasks, if each task is completely independent
of the others, then it is a prime candidate for executing those tasks in
parallel, each on its own core.

\url{https://nceas.github.io/oss-lessons/parallel-computing-in-r/parallel-computing-in-r.html}

Example: iris data with 5 columns
\begin{verbatim}
> iris
    Sepal.Length Sepal.Width Petal.Length Petal.Width    Species

1
2
.
.
.
150
\end{verbatim}

Example:
\begin{verbatim}
# return row indices in that the fifth-column
which(iris[,5] != "setosa")

# ... then extract 1st and 5th columns on those rows
x <- iris[which(iris[,5] != "setosa"), c(1,5)]

\end{verbatim}


Example: multiple tasks in serial
\begin{verbatim}

trials <- 10000

res <- data.frame()


system.time({
  trial <- 1
  while(trial <= trials) {
    ind <- sample(100, 100, replace=TRUE)
    result1 <- glm(x[ind,2]~x[ind,1], family=binomial(logit))
    r <- coefficients(result1)
    res <- rbind(res, r)
    trial <- trial + 1
  }
})
\end{verbatim}

\section{lapply() versions - faster than loop}

In R, using \verb!apply()! is often significantly faster than the equivalent
code in a loop. To do that, we need to convert our task to a function, and then
use the \verb!*apply()! family of R functions to apply that function to all of
the members of a set.
\begin{verbatim}

trials <- seq(1, 10000)  #MUST BE A SEQUENCE NOW

system.time({

  results <- lapply(trials, boot_fx)
})

\end{verbatim}

\section{mcapply() versions - faster than loop}

Use the multiple cores on a local computer through mclapply,
which is analogous to lapply, but distributes the tasks to multiple processors.


The parallel library can be used to send tasks (encoded as function calls) to
each of the processing cores on your machine in parallel.

mclapply gathers up the responses from each of these function calls, and returns
a list of responses that is the same length as the list or vector of input data
(one return per input item).



\section{makeCluster, clusterApply}

Use multiple processors on local (and remote) machines using makeCluster and clusterApply

\begin{verbatim}
In this approach, one has to manually copy data and code to each cluster member using clusterExport

This is extra work, but sometimes gaining access to a large cluster is worth it

\end{verbatim}



\section{foreach library}


\section{doParallel library}



%%% Local Variables: 
%%% mode: latex
%%% TeX-master: "R_language"
%%% End: 

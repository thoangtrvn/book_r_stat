\chapter{Heteroscedasticity - Homo}
\label{chap:heteroscedasticity}

Even though {\it normal distribution} is the common assumption before applying
the given test statistics, most of the methods of detecting heteroscedasticity
outlined below can be modified for use even when the data do not come from a
normal distribution. In many cases, this assumption can be relaxed, yielding a
test procedure based on the same or similar test statistics but with the
distribution under the null hypothesis evaluated by alternative routes: for
example, by using asymptotic distributions which can be obtained from asymptotic
theory, or by using resampling.

NOTE: Truely binomially distributed data is heteroscedastic. With the true mean
$p$ and variance $p(1-p)$, when we collect data and $p$ varying over the data,
then the expected observation errors is a function of the number of
observations. This causes the problem when  fitting and when attempting to
interpret coefficient significances. A first pass fix is to apply {\it
stabilizing transform}, e.g. sqrt(), arcsinh(), to make the observed variance
nearly constant over a wide range of $p$.
\footnote{\url{http://biomet.oxfordjournals.org/content/35/3-4/246.extract}}



\url{http://en.wikipedia.org/wiki/Heteroscedasticity}


\section{Park test}


\section{Glejser test}


\section{White test}


\section{Breusch-Pagan test}


\section{Goldfeld-Quandt test}

\section{Cook-Weisberg test}

\section{Harrison-McCabe test}

\section{Brown-Forsythe test}

\section{Levene test}


\section{Groups of data}

\subsection{F-test of equality of variances}

\subsection{Cochran's C test}

\subsection{Hartley's test}


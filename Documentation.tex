%%
%% Documentation.tex
%% Login : <hoang-trong@hoang-trong-laptop>
%% Started on  Mon Jun 22 21:42:54 2009 Hoang-Trong Minh Tuan
%% $Id$
%% 
%% Copyright (C) 2009 Hoang-Trong Minh Tuan

\chapter{Document your code}
\label{chap:document-your-code-1}



This is a very important part if you want your code to be useful to
others or later use.  Suppose that you have written your own function
mypow
\begin{lstlisting}
  mypow <- function(x, power) x^power
\end{lstlisting}

Now, you can use the command {\bf prompt(func\_name)} to document your function

\begin{lstlisting}
> prompt (mypow)
\end{lstlisting}
You will see something like this\footnote{\url{http://www.hsph.harvard.edu/biostats/courses/individual/bio271/lectures/L6/L6.pdf}}.
\begin{verbatim}
created file named mypow.Rd in the current directory
  Edit the file and move it to the appropriate directory, 
possibly /usr4/biostatistics/src/R-1.4.1/src/library/<pkg>/man/
\end{verbatim}


%%% Local Variables: 
%%% mode: latex
%%% TeX-master: "R_language"
%%% End: 

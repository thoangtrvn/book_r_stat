
\chapter{Statistics tests (Categorical data - 2 samples)}
\label{chap:stat-tests-categ}

In the previous chapters, data are {\bf cardinal}. Now, we will learn
testing methods working on {\bf ordinal data}.

\begin{itemize}
\item cardinal data:
  \begin{itemize}
  \item if the zero is arbitrary, we have {\bf interval scale},
    e.g. temperature (Kelvin, Celcius)
  \item if the zero is fixed, we have {\bf ratio scale}, e.g. height measure
  \end{itemize}

\item ordinal data: those that can be ordered, but the magnitude of
  difference is no meaningful
  \begin{itemize}
  \item categorical data, e.g. data is divided into groups of 1-4,
    5-7, 8-10; or cancer classification (normal, mild, moderate,
    severe), or drug treatment results( better, normal, worse) 
  \item when a number is assigned to represent a categorical data,
    e.g. 1=better, 2= normal, 3 = worse
  \end{itemize}

\item nominal data: data values can be classified into categories but
  the categories have no specific ordering
  \begin{itemize}
  \item disease names
  \end{itemize}
\end{itemize}

\section{Sign test}
\label{sec:sign-test}

Suppose we have two samples of ordinal data
\begin{itemize}
\item $x_i$ - degree of redness on the A arm
\item $y_i$ - degree of redness on the B arm
\end{itemize}
We want to test whether the degree of redness in A is larger than in B
arm or not?

\subsection{Normal-theory (large sample)}
\label{sec:normal-theory-large}


\subsection{Exact method}
\label{sec:exact-method}




%%% Local Variables: 
%%% mode: latex
%%% TeX-master: "R_language"
%%% End: 

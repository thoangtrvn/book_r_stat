\chapter{Time Series}


A {\it r vector-valued time series} is a r vector-value function
written in the form
\begin{equation}
\left[  \begin{array}{c}
X_1(t) \\
X_2(t)\\
\ldots \\
X_r(t)
\end{array}\right]
\end{equation}
each component get real-valued, $t$ represents the time, taking on the value
0, $\pm 1$, $\pm 2$, \ldots

Time series data occurs in both physical sciences, and social sciences.

\section{Multi-channel data}

Reference: Mitra - Pesaran (1999)
\begin{enumerate}
  \item multielectrode recordings, 
  
  \item optical brain images using intrinsic (Blasdel and Salama, 1986; Grinvald
  et al., 1992) or extrinsic (Davila et al., 1973) contrast agents, 
  
  \item functional magnetic resonance imaging (fMRI) (Ogawa et al., 1992; Kwong
  et al., 1992), and 
  
  \item magnetoencephalography (MEG) (Hamalainen et al., 1993)
\end{enumerate}

Techniques for treating multivariate statistics in the field of pattern
recognition are well developed; yet those for time series data are less well
developed; except in special cases.

\section{Multivariate time series analysis}
\label{sec:multivariate-time-series-analysis}

Time series analysis is about the study of data collected through time.
\begin{enumerate}
  \item  stationary time series

  \item globally non-stationary (e.g. in financial time series)
\end{enumerate}


\section{-- Multitaper spectral method}
\label{sec:Multitaper-spectral-method}

This is the framework for performing spectral analysis of univariate and
multivariate time series which is good for the condition of 
\begin{itemize}
  \item very short data segments, e.g. in fMRI (due to the cost of collecting)
  
  \item presence of non-stationary data
\end{itemize}

\section{General Linear Model (GLM)}
\label{sec:GLM}

The general linear model (GLM) is developed based on regression and correlation
methods.
\url{http://www.statsoft.com/Textbook/General-Linear-Models}

The general linear model can be seen as an extension of linear multiple
regression for a single dependent variable.
Understanding the multiple regression model
(Sect.\ref{sec:multiple-regression-model}) is fundamental to understanding the
general linear model.


\subsection{History}

The emergence of the theory of algebraic invariants in the 1800's that made the
general linear model, as we know it today, possible.
The theory of algebraic invariants developed from the groundbreaking work of
19th century mathematicians such as Gauss, Boole, Cayley, and Sylvester.
{\bf it seeks to identify those quantities in systems of equations which remain
unchanged under linear transformations of the variables in the system}.

Eigenvalues, eigenvectors, determinants, matrix decomposition methods; all
derive from the theory of algebraic invariants.

Example of (invariant property):  The correlation coefficients between two
variables is unchanged by linear transformations of either or both variables.

If we did not have statistics that are invariant to the scaling of the variables
involved, the development of useful statistical techniques would be nigh
impossible.

From the theory of algebraic invariants:
\begin{enumerate}
  \item  the development of the linear regression model in the late 19th
  century,
  
  \item the development of correlational methods shortly thereafter
  
  \item Regression and correlational methods, in turn, serve as the basis for
  the general linear model. (Sect.\ref{sec:GLM})

The general linear model can be seen as an extension of linear multiple
regression for a single dependent variable.
\end{enumerate}






\chapter{Regression methods}

\section{Multiple regression models}
\label{sec:multiple-regression-model}

The purpose of multiple regression, the computational algorithms used to solve
regression problems, and how the regression model is extended in the case of the
general linear model.
\url{http://www.statsoft.com/Textbook/Multiple-Regression}







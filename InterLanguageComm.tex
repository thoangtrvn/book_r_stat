%%
%% InterLanguageComm.tex
%% Login : <hoang-trong@hoang-trong-laptop>
%% Started on  Thu Jul 16 17:34:44 2009 Hoang-Trong Minh Tuan
%% $Id$
%% 
%% Copyright (C) 2009 Hoang-Trong Minh Tuan
%%

\chapter{Inter-Language Communication}
\label{chap:inter-lang-comm}


To make a call to compiled code that has been loaded to R, you can
use\footnote{\url{http://hosho.ees.hokudai.ac.jp/~kubo/Rdoc/library/base/html/Foreign.html}}.
\begin{lstlisting}
.C(name, ..., NAOK = FALSE, DUP = TRUE, PACKAGE, ENCODING)
.Fortran(name, ..., NAOK = FALSE, DUP = TRUE, PACKAGE, ENCODING)
.External(name, ..., PACKAGE)
    .Call(name, ..., PACKAGE)

.External.graphics(name, ..., PACKAGE)
    .Call.graphics(name, ..., PACKAGE)
\end{lstlisting}

\section{Load a library}
\label{sec:load-library}

Here, you will know how to load a DLL (shared library) written in
another language (C, Fortran) into R. You use {\bf dyn.load()}
function\footnote{\url{http://hosho.ees.hokudai.ac.jp/~kubo/Rdoc/library/base/html/dynload.html}}.
\begin{lstlisting}
dyn.load(x, local = TRUE, now = TRUE, ...)
dyn.unload(x)

is.loaded(symbol, PACKAGE = "", type = "")
\end{lstlisting}


%%% Local Variables: 
%%% mode: latex
%%% TeX-master: "R_language"
%%% End: 

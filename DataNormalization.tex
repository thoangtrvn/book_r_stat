%%
%% DataNormalization.tex
%% Login : <hoang-trong@hoang-trong-laptop>
%% Started on  Sun Sep 20 22:02:05 2009 Hoang-Trong Minh Tuan
%% $Id$
%% 
%% Copyright (C) 2009 Hoang-Trong Minh Tuan
%%

\chapter{Data normalization}
\label{chap:data-normalization}

Data analysis involves three major tasks:
\begin{enumerate}
\item point estimation

\item hypothesis testing

\item confidence interval
\end{enumerate}


When performing data analysis and statistical comparison between
different data sets, it is more common belonging to hypothesis
testing. Thus, it's important that these data are normalized to the
same of some statistical summaries (e.g. mean, variance, median,
normal distribution...). In this chapter, we will discuss some
techniques to normalize the data.

\section{Normal distribution}
\label{sec:normal-distribution}

Many of statistical analysis techniques assume the normal distribution
of data. So, rescaled the data so that the new data points are
redistributed in a look-like normal distribution is strongly desired. 

This task is problem-dependent. However, log transform is mainly
used. One major reason is that the new values are smaller so that it
can fit to the graph axes. 

\subsection{Microarray}
\label{sec:microarray}


In microarray data, it is more common to use $\log_2$
transformation. Log2 transformation cause the data to be more normal
without changing the relationship between values. 


\section{Median corrections}
\label{sec:median-corrections}


To make all data have the same median value, e.g. zero, it is first to
find the median from each data set, the sweep those median out from
the data sets. 

{\bf Example}: X is an array whose columns are different data
set. Each column has a different median.

\begin{lstlisting}
X_med = apply(X, 2, median)
new_X = sweep(X, 2, X_med)
\end{lstlisting}

\section{Mean corrections}
\label{sec:mean-corrections}

This can be done similar to that of median corrections.

\section{Variance corrections}
\label{sec:variance-corrections}



%%% Local Variables: 
%%% mode: latex
%%% TeX-master: "R_language"
%%% End: 

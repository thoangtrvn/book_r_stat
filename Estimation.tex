
\chapter{Estimation}
\label{chap:estimation}

There are two types of estimation:
\begin{enumerate}
\item Point estimation (mean, standard deviation...)
\item Interval estimation (confidence interval, range, ...)
\end{enumerate}
and we start with
\begin{enumerate}
\item {\bf Deductive}: (deductive, i.e. ask something about that
  specific sample or population only)
  \begin{itemize}
  \item A single sample 
  \item A single population
  \end{itemize}
\item {\bf Inductive}: (inductive, i.e. ask something about the
  population that can be inferred from the given random sample)
  \begin{itemize}
  \item A single sample with given distribution 
  \item A single sample with unknown distribution
  \end{itemize}
\end{enumerate}

For detail, read Sect.~\ref{sec:stat-funct} for deductive,  and
Sect.~\ref{chap:generate-data} for inductive. 
\section{Point estimation}
\label{sec:point-estimation}

\subsection{Population mean}
\label{sec:population-mean}


\subsection{Population variance}
\label{sec:population-variance}

\begin{equation*}
  \sigma^2 = \frac{\sum_{i=1}^n (x_i-\bar{x})^2}{n-1}
\end{equation*}

\begin{lstlisting}
X = rnorm(10, size = 100)
Y1 = sample(x, 100, rep = F)
Y2 = sample(x, 100, rep = F)
...
Y10 = sample(x, 100, rep = F)
Pop_mean = mean(var(Yi))
\end{lstlisting}

\subsection{Central Limit Theorem (CLT)}
\label{sec:centr-limit-theor}

If we take $n$ random samples $X_1,X_2,\dots, X_n$ from a population
with known mean $\mu$ and variance $\sigma^2$, then when $n$ is large
enough, the sample mean $\bar{X}$ follow normal distribution with mean
is the population mean, and variance is the variance of the population
divided by $n$; even the individuals in the population doesn't follow
normal distribution.
\begin{equation}
  \label{eq:27}
  \bar{X} =  E[X]  \sim \text{Norm}(\mu,\frac{\sigma^2}{n})
\end{equation}
Then, 95\% of all such samples has the sample means fall within the
interval 
\begin{equation}
  \label{eq:37}
  (\mu- z_{1-\alpha} \sigma/\sqrt{n}, \mu+z_{1-\alpha} \sigma/\sqrt{n})
\end{equation}
with $\alpha=1-95\%=0.05$ and $z_{0.05}=1.96$. This is known as the
{\bf confidence interval}. The lower and upper boundary correspond to
the 2.5-th and 97.5-th percentiles from the normal distribution. 

We can easily convert $\bar{X}$ to standard normal distribution, by
changing to a new r.v. Z
\begin{equation}
  \label{eq:28}
  Z = \frac{\bar{X}-\mu}{\sigma/n}
\end{equation}
with $Z \sim \text{Norm}(0,1)$. However, in practice, we don't know
population standard deviation $\sigma$. Using sample standard
deviation, we come up with the r.v. that follow the so-called
Student-$t$ distribution with (n-1) degree of freedom (read
Sect.~\ref{sec:stud-t-distr}).
\begin{equation}
  \label{eq:29}
  Z' = \frac{X-\mu}{s/n} \sim \text{t}_{(n-1)}
\end{equation}


%%% Local Variables: 
%%% mode: latex
%%% TeX-master: "R_language"
%%% End: 

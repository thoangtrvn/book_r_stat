%%
%% ValidateData.tex
%% Login : <hoang-trong@hoang-trong-laptop>
%% Started on  Sun Jun 14 16:39:43 2009 Hoang-Trong Minh Tuan
%% $Id$
%% 
%% Copyright (C) 2009 Hoang-Trong Minh Tuan
%% This program is free software; you can redistribute it and/or modify
%% it under the terms of the GNU General Public License as published by
%% the Free Software Foundation; either version 2 of the License, or
%% (at your option) any later version.
%% 
%% This program is distributed in the hope that it will be useful,
%% but WITHOUT ANY WARRANTY; without even the implied warranty of
%% MERCHANTABILITY or FITNESS FOR A PARTICULAR PURPOSE.  See the
%% GNU General Public License for more details.
%% 
%% You should have received a copy of the GNU General Public License
%% along with this program; if not, write to the Free Software
%% Foundation, Inc., 59 Temple Place, Suite 330, Boston, MA 02111-1307 USA
%%


\chapter{Validate Data}
\label{chap:validate-data}

\section{NaN, NA}
\label{sec:nan-na}

\subsection{NA}
\label{sec:na}


Missing values in the statistical sense are values that not known, NA.
The default type of NA is {\it logical}. So, the occurrence of NA may
trigger a logical rather than a numerical indexing. However, it may get coerced
to another type.

\begin{lstlisting}
if (na.rm)

        x <- x[!is.na(x)]

    else if (any(is.na(x)))
        stop("Missing values and NaN's not allowed if `na.rm' is FALSE")
    if (any((p.ok <- !is.na(probs)) & (probs < 0 | probs > 1))) 
        stop("probs outside [0,1]")
    if (na.p <- any(!p.ok)) {
        o.pr <- probs
        probs <- probs[p.ok]
\end{lstlisting}

Any numerical or logical calculation with NA generally return NA.

NA is not comparable to any values, including itself. So, to test for
the missing value of an object or entry, {\bf is.na()} function is
used. However, in matching, a NA value will match a NA value.

(since R 1.5.0) A string `NA' is different from a NA character
type. So, to specify a NA character type, use
\begin{lstlisting}
as.character(NA)
\end{lstlisting}

(since R 2.5.0) To use a NA value of appropriate type, there are some
new constants: \lstinline!NA_integer_, NA_real_, NA_complex_,!
\lstinline!NA_character_!.


\subsection{NaN}
\label{sec:nan}

Numerical calculation whose result is undefined, e.g. $0/0$, produce
the value NaN. To check for an object/expression is NaN, {\bf
  is.nan(object)} is used, {\bf is.na(NaN)} return TRUE. 

Coercing NaN to integer or logical type result in a NA value. However
coercing it to a character type result in a `NaN` string. 

Any comparison with NaN value result in a NA value (logical). 

A NaN value only match to another NaN value, not NA.



%%% Local Variables: 
%%% mode: latex
%%% TeX-master: "R_language"
%%% End: 

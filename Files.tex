%%
%% Files.tex
%% Login : <hoang-trong@hoang-trong-laptop>
%% Started on  Fri Jun 12 19:40:46 2009 Hoang-Trong Minh Tuan
%% $Id$
%% 
%% Copyright (C) 2009 Hoang-Trong Minh Tuan
%% This program is free software; you can redistribute it and/or modify
%% it under the terms of the GNU General Public License as published by
%% the Free Software Foundation; either version 2 of the License, or
%% (at your option) any later version.
%% 
%% This program is distributed in the hope that it will be useful,
%% but WITHOUT ANY WARRANTY; without even the implied warranty of
%% MERCHANTABILITY or FITNESS FOR A PARTICULAR PURPOSE.  See the
%% GNU General Public License for more details.
%% 
%% You should have received a copy of the GNU General Public License
%% along with this program; if not, write to the Free Software
%% Foundation, Inc., 59 Temple Place, Suite 330, Boston, MA 02111-1307 USA
%%

\chapter{Files}
\label{chap:files}

Reading data into a statistical analysis system and export the result
to some other system are essential tasks.  In general, a data set is
organized in column-wise. As a matter of fact, data file should also
be organized in a format such that elements can be detected easily and
organized into columns.

If the file is a text file, each row is a record. In a single line,
there is a delimiter (tab, whitespace...)  to separate elements in a
row. Tab-separated and whitespace-separated text file is an important
part of working with R. To read/write to these text files, R provides
2 functions: {\bf read.table()} and {\bf write.table()}.

R is not really good in manipulating large-scale data. Thus, it is
recommended to use the facilities provided by other languages,
e.g. Java, Perl or Python, and then make use of the packages written
in these languages from R. Some of them includes: SJava, RSPerl,
RSPython packages from \href{http://www.omegahat.org}{OmegaHat}
project; and rJava from CRAN.

% R is not well suit for large-scale data. Thus, there are several
% packages (SJava, RSPerl, RSPython - from Omegahat project, and rJava -
% from CRAN) that allows user to integrate Python, Perl, Java... codes
% into R.

Normally, the files are {\it tab-separated} or {\it
  whitespace-separated} text files.  

\section{Excel file}

For writing to Excel file, it's best to use \verb!openxlsx! as it is not Java-depdendent.

For reading Excel file, it is best to use  \verb!readxl! package

\subsection{openxlsx}

 Through the use of 'Rcpp', read/write times are comparable to the 'xlsx' and
 'XLConnect' packages with the added benefit of removing the dependency on Java.
 
\url{https://cran.r-project.org/web/packages/openxlsx/}


\subsection{readxl package}
\label{sec:readxl-package}

Readxl and Rcpp are needed to load an .xls or .xlxs file.

\begin{verbatim}
> install.package(readxl)
# or
install.packages("tidyverse")

> library(readxl)
> readxl_example()
 [1] "clippy.xls"    "clippy.xlsx"   "datasets.xls"  "datasets.xlsx"
 [5] "deaths.xls"    "deaths.xlsx"   "geometry.xls"  "geometry.xlsx"
 [9] "type-me.xls"   "type-me.xlsx" 
\end{verbatim}


\section{Import}
\label{sec:import}

\subsection{into a vector or list}
\label{sec:into-vector-or}

You can read data from a file into a vector or a list using
{\bf scan()} command. The default is {\it whitespace-separated}
file. You can tell a different delimiter using {\it sep}
parameter. You can also skip a number of lines before starting to read
data by using {\it skip} parameter. You can also tell the maximum
number of lines of data to be read using {\it nlines} parameter.

\begin{lstlisting}
 scan(file = "", what = double(0), nmax = -1, n = -1, sep = "",
          quote = if(identical(sep, "\n")) "" else "'\"", dec = ".",
          skip = 0, nlines = 0, na.strings = "NA",
          flush = FALSE, fill = FALSE, strip.white = FALSE,
          quiet = FALSE, blank.lines.skip = TRUE, multi.line = TRUE,
          comment.char = "", allowEscapes = FALSE,
          encoding = "unknown")
\end{lstlisting}
{\it what} tells the type of data to be read (which can be either
'logical', 'integer', 'numeric', 'complex', 'character', 'raw' and
'list').

{\bf IMPORTANT}: If {\it what} is a 'list', then each line is a record
which contains length, fields and the list components which must be of
either 6 types listed.

{\it nmax} is the maximum number of values to read. By default, it
reads all. 

{\bf IMPORTANT}: If {\it what} is a 'list', then {\it nmax} tells the
number of lines (records) to read. 

\subsection{into a data.frame}
\label{sec:into-data.frame}

The data from the file is organized in table format, i.e. it can have
a header row, and row names. Thus, we have to use {\bf read.table()}
command. 

\begin{lstlisting}
read.table(file, header = FALSE, sep = "", 
           quote = "\"'", dec = ".", 
           row.names, col.names,
           as.is = !stringsAsFactors,
           na.strings = "NA", colClasses = NA, 
           nrows = -1, skip = 0, check.names = TRUE, 
           fill = !blank.lines.skip,
           strip.white = FALSE, 
           blank.lines.skip = TRUE,
           comment.char = "#",
           allowEscapes = FALSE, flush = FALSE,
           stringsAsFactors = default.stringsAsFactors(),
           encoding = "unknown")
\end{lstlisting}
By default: there is no header row (header=FALSE), the file is {\it
  whitespace-separated}. If there is a column read as character
string, we need to specify the quoting character using {\it quote}
parameter. 

As the data is in table format, it can have row names which can be
specified through a vector (using {\it row.names} assigned to a
vector), a number which tells the columns in the file that contains
row names (using {\it row.names} assigned to a column number), a
character string which is the name of a table column whose elements
are row names (using {\it row.names} assigned to a character string,
and {\it header}=TRUE). If {\it header}=TRUE, and the number of fields
in the first row is less than the number of columns in the data, then
the first column in the data is designated for row names. Otherwise,
row names are automatically numbered with 1, 2, 3.... 


\section{Export to a text file}
\label{sec:export-text-file}

The primary function is {\bf cat()}. 


\section{Export to a binary file}
\label{sec:export-binary-file}


\url{http://cran.r-project.org/doc/manuals/R-data.html}

%%% Local Variables: 
%%% mode: latex
%%% TeX-master: "R_language"
%%% End: 

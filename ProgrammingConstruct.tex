
\chapter{Programming constructs}
\label{chap:progr-constr}


\section{Control statements}
\label{sec:control-statements}

FUNCTION
\begin{lstlisting}
function (arg_list) {
  expr

  return (value)
}
\end{lstlisting}


CONDITION
\begin{lstlisting}
if (cond) {
  cons.expr
}else{
  alt.expr
}
\end{lstlisting}

WHILE
\begin{lstlisting}
repeat {
  expr

  break      # to terminate the loops

  next       # skip the next statements and jump to the beginning of
             # the loop

}
\end{lstlisting}


FOR
\begin{lstlisting}
for (name in expr) {
 exprs

}        # name : the loop variable
         # expr is often a sequence, created by using seq() function
         # or using the form start:end structure

tapply(1:n, fac, sum) # apply function 'sum' to each comb of factor
       # levels 

\end{lstlisting}

Example:
\begin{lstlisting}
muy = 120
sigma = 24

n = 100
x = matrix(0, nrow = n, ncol = n)
for (i in 1:n) {
  set.seed(i)
  x[i,] = rnorm(100, mean = muy, sd = sigma)
}
\end{lstlisting}


\section{Data types}
\label{sec:data-types}


%%% Local Variables: 
%%% mode: latex
%%% TeX-master: "R_language"
%%% End: 
